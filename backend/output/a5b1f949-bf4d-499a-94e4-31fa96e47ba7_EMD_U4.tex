
\documentclass{article}
\usepackage{amsmath}
\usepackage{graphicx}

\begin{document}

Analog and Digital signals
Analog and digital signals are the types of signals carrying information. The major
difference between both signals is that the analog signals have continuous
electrical signals, while digital signals have non-continuous electrical signals.
Analog signals were used in
many systems to produce
signals
to
carry
information. These signals
are
continuous
in
both
values and time. A human
voice, analog phones, and
thermometer are some of
the
examples
of
analog
signals.
Unlike analog signals, digital signals are not continuous, but signals are discrete in
value and time. These signals are represented by binary numbers and consist of
different voltage values. Digital signals do not produce noise. Digital computers
and digital phones are some of the examples of digital signals. Long-distance
transmission: Because digital signals can be sent over greater distances without
considerable signal deterioration, they are ideal for long-distance communication.


Difference between Analog and Digital Signals
Analog Signals
Digital Signals
Continuous signals
Discrete signals
Represented by sine waves
Represented by square waves
Human voice, natural sound, 
analog electronic devices are a 
few examples
Computers, optical drives, and 
other electronic devices
Continuous range of values
Discontinuous values
Records sound waves as they 
are
Converts into a binary 
waveform
Only used in analog devices
Suited for digital electronics 
like computers, mobiles and 
more


Digital Circuits
Digital Signals
A Digital Signal is a type of Signal that has two discrete levels, either HIGH (1) or
LOW (0). These two levels are usually represented by-
LOGIC 1 = HIGH = TRUE = ON = YES
LOGIC 0 = LOW = FALSE = OFF = NO
The concept of the Binary number system is the accurate representation of
Digital Signals. Digital Signals work on the principles of Boolean Algebra, binary
mathematics developed by George Boolean.
Digital Circuits
Digital Signals operate at high speeds and drive Digital Circuits that contain some
basic components such as Diodes, Inductors, Capacitors, Resistors, Batteries,
and Logic Gates.
There are many Logic families that follow the principle of Digital Signals.
Examples of such Logic families consider 3.5V to 5V voltage as high logic and 0V
to 1V as low logic. This means voltage lying anywhere between 3.5V to 5V would
be represented by 1 and voltage lying anywhere between 0V to 1V would be
represented by 0. The actual value of voltage is not important in digital signals.


Digital Circuits
Representation for a range of voltages as 1 or 0, makes digital circuit operation
simpler than analog circuits. Operating only in two states, either high or low,
makes these signals fast, and less susceptible to noise, temperature, and
irrespective of the aging components.
As compared to analog systems, Digital circuits follow the concepts of electrical
network analysis and have a “memory”.
There are two types of Digital Circuits: Combinational Digital circuits and
Sequential Digital Circuits.
Combinational Digital Circuits are the type of digital circuits in which output
depends upon inputs at that present time.
Sequential Digital Circuits are time-dependent digital circuits in which the
outputs depend upon past states as they have memory units.


Binary Number System
In the Binary Number System, we have two states “0” and “1” and these two
states are represented by two states of a transistor. If the current passes through
the transistor then the computer reads “1” and if the current is absent from the
transistor then it read “0”. Thus, alternating the current the computer reads the
binary number system. Each digit in the binary number system is called a “bit”.
Binary Number System
Binary Number System is the number system in which we use two digits “0” and
“1” to perform all the necessary operations. In the Binary Number System, we
have a base of 2. The base of the Binary Number System is also called the radix of
the number system.
In a binary number system, we represent the number as,
•(11001)2
In the above example, a binary number is given in which the base is 2. In a binary
number system, each digit is called the “bit”. In the above example, there are 5
digits.


Decimal Number
Binary Number
Decimal Number
Binary Number
1
001
11
1011
2
010
12
1100
3
011
13
1101
4
100
14
1110
5
101
15
1111
6
110
16
10000
7
111
17
10001
8
1000
18
10010
9
1001
19
10011
10
1010
20
10100


Binary to Decimal Conversion
A binary number is converted into a decimal number by multiplying each digit of
the binary number by the power of either 1 or 0 to the corresponding power of
2. Let us consider that a binary number has n digits, B = an-1…a3a2a1a0. Now, the
corresponding decimal number is given as
D = (an-1 × 2n-1) +…+(a3 × 23) + (a2 × 22) + (a1 × 21) + (a0 × 20)
Example: Convert (10011)2 to a decimal number.
Solution:
The given binary number is (10011)2.
(10011)2 = (1 × 24) + (0 × 23) + (0 × 22) + (1 × 21) + (1 × 20) = 16 + 0 + 0 + 2 + 1 = 
(19)10
Hence, the binary number (10011)2 is expressed as (19)10.


Decimal to Binary Conversion
A decimal number is converted into a binary number by dividing the given
decimal number by 2 continuously until we get the quotient as 1, and we write
the numbers from downwards to upwards.
Example: Convert (28)10 into a binary number.
Solution:
Hence, (28)10 is expressed as (11100)2.


\end{document}
